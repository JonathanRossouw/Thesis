\documentclass[b]{beamer}
 \usepackage{tikz}
\usepackage{graphicx}
\usepackage{textgreek}

\usepackage{subfigure}
\usepackage{amsmath,amsthm}
\usepackage{beamerthemesplit}
\usepackage[greek, english]{babel}
\usepackage{ tipa }

\useoutertheme{shadow}
\usepackage{amssymb}
\usepackage{color}
 \usepackage{mathrsfs}
 \usepackage{multirow,array}
\usetikzlibrary{arrows,shapes,backgrounds}
\tikzstyle{every picture}+=[remember picture]
\tikzstyle{na} = [baseline=-.5ex]


\makeatother
\pagenumbering{alph}

\newcommand*\circled[1]{\tikz[baseline=(char.base)]{
  \node[shape=circle,draw,inner sep=2pt] (char) {#1};}}

%	\usetikzlibrary{arrows}
%	\usetikzlibrary{shapes}
	\newcommand{\mymk}[1]{%
		\tikz[baseline=(char.base)]\node[anchor=south west, draw,rectangle, rounded corners, inner sep=2pt, minimum size=7mm,
		text height=2mm](char){\ensuremath{#1}} ;}

\renewcommand{\thesubsubsection}{\thesubsection.\alph{subsubsection}}

\mode<presentation>
{
\usetheme{Warsaw}
}
\author[Tina Koziol]{ Tina Koziol
}
 \institute{
 University of Cape Town
}
\title[ Reserve Bank Workshop] {Asset fire-sales, contagion and systemic risk - \\Evidence from the South African banking sector}
\date{11 December 2016}

\setbeamercolor{description item}{fg=darkred!80!black}
\setbeamercolor{footnote}{fg=orange}
\setbeamercolor{date}{fg=darkblue}

%\setbeamercolor{normal text}{bg=ZurichOrange!30!}
\setbeamercolor{author}{fg=darkblue!80!}
\definecolor{darkred}{rgb}{0.8,0,0}
\setbeamercolor{item}{fg=blue}
\definecolor{darkblue}{rgb}{0,0,0.8}
\setbeamercolor{itemize/enumerate body}{fg=black}
\setbeamercolor{itemize/enumerate subbody}{fg=darkblue}
\setbeamercolor{itemize/enumerate subsubbody}{fg=green!25!black}

\newcommand{\blackcolors}{\setbeamercolor{itemize/enumerate body}{fg=black}
\setbeamercolor{itemize/enumerate subbody}{fg=black}
\setbeamercolor{itemize/enumerate subsubbody}{fg=black}}
\setbeamercolor{block body}{bg=blue!20,fg=black}

\begin{document}
\maketitle

\frame{\frametitle{Motivation}
\begin{itemize}
\item Growing interconnectedness in the banking sector
\begin{itemize}
\item Upside: Diversification and risk sharing
\vspace{2mm}
\item Downside: Intensifying amplification of shocks and contribution to systemic risk
\end{itemize}

\item How to measure and quantify systemic risk?
\begin{itemize}
\item Which bank contributes most?
\item Which assets?
\end{itemize}
 \end{itemize}
 }
 %%%%%%%%%%%%%%%%%%%%%%%%%%%%%%%%%%%%%%%%%%%%%%%%%%%%%%%%%%%%%%%%%%%%%%%%%%%

\frame{\frametitle{Outline}
\begin{itemize}
\item Systemic risk and fire-sale spillovers - a model of asset fire-sale propagation for South African financial system

% \bigskip
% \item
% \bigskip
% \item
% \bigskip
\bigskip
\begin{itemize}
\item \large Model framework
\bigskip
\item \large Results
\end{itemize}
\bigskip
\item BlackRhino
\bigskip
\item Data on Capital flows
\begin{itemize}
\bigskip
\item \large Morning Star \& ASISA
% \bigskip
\end{itemize}
 \end{itemize}
 }
 %%%%%%%%%%%%%%%%%%%%%%%%%%%%%%%%%%%%%%%%%%%%%%%%%%%%%%%%%%%%%%%%%%%%%%%%%%%
\frame{\frametitle{Literature on systemic risk}
\begin{itemize}
        \item Systemicness: Size? Leverage? Connectedness?
% * <tinakoziol@yahoo.de> 2016-12-10T09:37:53.412Z:
%
% ^.
        \smallskip
		\begin{enumerate}
		\item \large Infer bank linkages from
correlations in market prices. \\
      	\textcolor{black}{CDS spread and bonds (Ang and Longstaff (2011), Giglio (2011))} \\
        \textcolor{black}{Comovement in equity returns (Adrian and Brunnermeier (2010), Acharya, Pedersen, Philippon and
Richardson (2010), Billio, Getmansky, Lo, and Pelizzon (2010), Diebold and Yilmaz (2011)
of financial intermediaries}
          \item \large   Study amplification from balance sheet linkages  \\
      	\textcolor{black}{De Haas and van Horen (2011),   }
        \item Simulations and scenario analysis (Arvai et al. (2009))
		\end{enumerate}
        \end{itemize}
       $ \rightarrow$ Implementation of asset fire-sales propagation model for the South African banking sector.
} %%%%%%%%%%%%%%%%%%%%%%%%%%%%%%%%%%%%%%%%%%%%%%%%%%%%%%%%%%%%%%%%%%%%%%%%%%%%%%%%%%%%%%%%%%%%%%%%%%%%%%%%
  \frame{\frametitle{Model overview }
- Banking sector model
- Shock propagation via fire-sales
Framework from Greenwood et al. (2015)
\begin{itemize}
		\item Initial shock: An initial exogenous shock hits the banking system. This can be a shock to one or several asset classes, or to equity capital.
		\item Direct losses: Banks holding the shocked assets suffer direct losses which lead to an increase in their leverage.
		\item Asset sales: In response to the losses, banks sell assets and pay off debt.
        \item Price impact: The asset sales have a price impact that depends on each asset’s
liquidity and the amount sold.
		\item Spillover losses: Banks holding the fire-sold assets suffer losses from spillovers
	\end{itemize}
} %%%%%%%%%%%%%%%%%%%%%%%%%%%%%%%%%%%%%%%%%%%%%%%%%%%%%%%%%%%%%%%%%%%%%%%%%%%%%%%%%%%%%%%%%%%%%%%%%%%%%%%%%
%%%%%%%%%%%%%%%%%%%%%%%%%%%%%%%%%%

 \frame{\frametitle{Fire-sale spillovers: Model set up}
 \begin{itemize}
	\item Banks $i = 1, ...,N$ and asset classes $k,...K$\vskip 0.1cm
	\item Bank $i$ has total assets $a_i$ with portfolio weight $m_{ik}$ on asset $k$ such that $\sum_k{m_{ik}} = 1$. On the liability side, bank $i$ has debt $d_i$ and equity capital $e_i$, resulting in leverage $b_i = e_i/d_i$.
 \end{itemize}
 ~\\
% For the whole banking system we have and $N \times N$ diagonal matrix of assets A with $A_{ii} = a_i$ an $N \times K$ matrix of portfolio weights $M$ with $M_{ik}=m_{ik}$ and $N \times N$ diagonal matrix of leverage ratios $B$ with $B_{ii} = b_i$ For the total system: $a = \sum_i{a_i}$,  $b = \sum_i{b_i}$,  $e = \sum_i{e_i}$ $d = \sum_i{d_i}$
\begin{enumerate}
\item Vector of asset returns $F = [f_1,...,f_k]$ leads to direct losses: \begin{center}
$\textcolor{red}{a_i\sum_k{m_{ik}}\textcolor{black}{f_k}} \mbox{ for bank i}$; $AMF$\ for the whole system $(\mathcal{I} \times \mathbf{1})$  \end{center}
%READ:: Assumption: banks sell assets and reduce debt to return to their initial leverage. To determine the shortfall a bank has to cover to get back to target leverage we multiply the loss by leverage b_i
\item What's the shortfall?\\
\begin{center} $\textcolor{red}{b_i}\textcolor{black}{a_i\sum_k{m_{ik}}{f_k}} \mbox{ for bank i}$; $\textcolor{red}{B}{AMF}$\ for the whole system $(\mathcal{I} \times \mathbf{1})$\end{center}
\item Banks raise this shortfall by selling assets proportionally to their weights $m_{ik}$ which leads to asset sales:\begin{center}
$\textcolor{red}{\sum_i{m_{ik'}}}\textcolor{black}{b_ia_i\sum_k{m_{ik}}{f_k}} \mbox{ for asset k'}$;
\vskip 0.2cm $\textcolor{red}{M'}{BAMF}$\ for the whole system$(\mathcal{K} \times \mathbf{1})$
\end{center}
\end{enumerate}}
%%%%%%%%%%%%%%%%%%%%%%%%%%%%%%%%%%%%%%%%%%%%%%%%%%%%%%%%%%%%%%%%%%%%%%%%%%%%%%%%%%%%%%%%%%%%%%%%%%%%%%%%%
%%%%%%%%%%%%%%%%%%%%%%%%%%%%%%%%%%%%%%%%%%%%%%%%%%%%%%%%%%%%%%%%%%%%%%%%%%%%%%%%%%%%%%%%%%%%%%%%%%%%%%%%%
%%%%%%%%%%%%%%%%%%%%%%%%%%%%%%%%%%

 \frame{\frametitle{Fire-sale spillovers: Model (2)}
\begin{itemize}
\item These asset sales have price impacts that depend on each asset’s illiquidity $l_k$. The illiquidity is measured in units of percentage points of price change per dollar amount sold which is standard in the empirical literature. Placing these illiquidity measures into a diagonal matrix L, the fire-sale price impacts are:
 \\
\begin{center}
$\textcolor{red}{l_k}{\sum_i{m_{ik'}}}\textcolor{black}{b_ia_i\sum_k{m_{ik}}{f_k}} \mbox{ for asset k'}$
\vskip 0.4cm
$\textcolor{red}{L}{M'BAMF}$\ for the whole system$(\mathcal{K} \times \mathbf{1})$
\end{center}
Price impacts cause spillover losses to all banks holding the assets that
were fire-sold which we can calculate as 1):\\
\begin{center}
$ \textcolor{red}{a_{i'}\sum_{k'}{m_{i'k'}}\textcolor{black}{l_k}}\textcolor{black}
{ \sum_i{m_{ik'}}b_ia_i\sum_k{m_{ik}}{f_k}} \mbox{ for bank i'}$
\vskip 0.4cm
$\textcolor{red}{AM}{LM'BAMF}$\ for the whole system$(\mathcal{I} \times \mathbf{1})$
\end{center}
\end{itemize}
}
%%%%%%%%%%%%%%%%%%%%%%%%%%%%%%%%%%%%%%%%%%%%%%%%%%%%%%%%%%%%%%%%%%%%%%%%%%%%%%%%%%%%%%%%%%%%%%%%%%%%%%%%%

%%%%%%%%%%%%%%%%%%%%%%%%%%%%%%%%%%%%%%%%%%%%%%%%%%%%%%%%%%%%%%%%%%%%%%%%%%%%%%%%%%%%%%%%%%%%%%%%%%%%%%%%%
%%%%%%%%%%%%%%%%%%%%%%%%%%%%%%%%%%%%%%%%%%%%%%%%%%%%%%%%%%%%%%%%%%%%%%%%%%%%%%%%%%%%%%%%%%%%%%%%%%%%%%%%%
%%%%%%%%%%%%%%%%%%%%%%%%%%%%%%%%%%

 \frame{\frametitle{Fire-sale spillovers: Model (3)}
\begin{itemize}
\item Summing the losses over all banks i′, total spillover losses suffered by the system {A, M, B, L} for a given initial shock F :
 \\

\begin{center}$\mathcal{L} =
\textcolor{red}{\sum_{i'}}{a_{i'}\sum_k{m_{i'k'}}\textcolor{black}{l_k}}\textcolor{black}
{ \sum_i{m_{ik'}}b_ia_i\sum_k{m_{ik}}{f_k}}  $
\vskip 0.4cm

 % $\textcolor{red}{1'}{AMLM'BAMF}$\ where 1' is a column vector of ones
\end{center}

$\mathcal{L}$ captures indirect losses suffered through spillovers, while direct losses are given by 1'AMF ($\sum_i{a_i\sum_k{m_{ik}f_k}}$)
\end{itemize}
$\Rightarrow$ from this we can derive measures of financial system vulnerability \\
\begin{enumerate}
\item Aggregate Vulnerability, is the fraction of system equity capital lost due to spillovers. $\mathcal{L}/e$
captures the aggregate vulnerability of the system to fire-sale spillovers
\item Systemicness of bank $i$: $ AV = \sum_i{S(i)}$.
  \item $S(i)$  is higher:
  the higher the leverage $b_i$,
  the higher connectedness  ($i$ owns illiquid and large assets hold by other banks), the bigger the bank  and the larger the shock \end{enumerate}

}
%%%%%%%%%%%%%%%%%%%%%%%%%%%%%%%%%%%%%%%%%%%%%%%%%%%%%%%%%%%%%%%%%%%%%%%%%%%%%%%%%%%%%%%%%%%%%%%%%%%%%%%%%

 \frame{\frametitle{Fire-sale spillovers: Implementation}
% The contribution to aggregate vulnerability by bank i is obtained by dropping the summation over i in the expression for aggregate vulnerability (2) which combines all banks’ individual asset sales into one total. It can also be interpreted as the aggregate vulnerability resulting from a shock only to bank
\begin{center}
\includegraphics[scale=0.65]{EQUITY_losses.png}
\end{center}
\begin{itemize}
\item 29 banks, 97\% of total assets of the banking system
\item Shock scenarios:
\begin{enumerate}
\item \large -30\% Household unsecured lending
\item \large -30\% Household mortgage lending

\item \large - 40\% Government bonds
\end{enumerate}

\end{itemize}

}

%%%%%%%%%%%%%%%%%%%%%%%%%%%%%%%%%%%%%%%%%%%%%%%%%%%%%%%%%%%%%%%%%%%%%%%%%%%%%%%%%%%%%%%%%%%%%%%%%%%%%%%%%

  \frame{\frametitle{Fire-sale spillovers: Implementation}
% The contribution to aggregate vulnerability by bank i is obtained by dropping the summation over i in the expression for aggregate vulnerability (2) which combines all banks’ individual asset sales into one total. It can also be interpreted as the aggregate vulnerability resulting from a shock only to bank
\begin{center}

\includegraphics[scale=0.54]{av.png}
\end{center}

}


%%%%%%%%%%%%%%%%%%%%%%%%%%%%%%%%%%
\frame{\frametitle{Methodology}
\begin{block}{Agent-based Modeling}

 ABM - novel approach which builds simulation models consisting of many  agents which engage with one another:
\begin{itemize}
\item  computational models of complex systems
\item use an individualistic approach (bottom-up)
\item simulate the systemic (emergent) effects caused by the actions and interactions between autonomous agents
\end{itemize}
For example: banks buy and sell securities/assets, take in deposits, take an investment strategy etc. \newline
\end{block}
} %%%%%%%%%%%%%%%%%%%%%%%%%%%%%%%%%%%%%%%%%%%%%%%%%%%%%%%%%%%%%%%%%%%%%%%%%%%%%%%%%%

\frame{\frametitle{Methodology}
How does this work?

\begin{enumerate}
\item Initialize a population of autonomous agents (objects) capable of making simple decisions in a domain. Agents can follow different (heterogeneous) or the same (homogeneous) strategy.
\item Create an environment in which the agents may interact with one another.
\item Feed new information to the model, allowing the agents to operate autonomously, and observe for emergent complex phenomena (e.g. bubbles, market crash).
\item Allow the agents to self-adapt to their changing environment through mechanisms such as machine learning or optimization algorithms.
\end{enumerate}
}


%%%%%%%%%%%%%%%%%%%%%%%%%%%%%%%%%%
\frame{\frametitle{Methodology}

\includegraphics[scale=0.35
]{Folie1.PNG}


}



%%%%%%%%%%%%%%%%%%%%%%%%%%%%%%%%%%



%%%%%%%%%%%%%%%%%%%%%%%%%%%%%%%%%%
\frame{\frametitle{}
Next steps:

\begin{itemize}
\item Introduce agent interactions and endogenous responses from market participants.
\item Finding an adequate agent based model to model systemic risk
~\\
\item Fire-sale model: calibration on SA data
\end{itemize}
} %%%%%%%%%%%%%%%%%%%%%%%%%%%%%%%%%%%%%%%%%%%%%%%%%%%%%%%%%%%%%%%%%%%%%%%%%%%%%%%%%%

\begin{frame}
	\begin{center}
    \Large{\textbf{South African Data}}
    \end{center}
\end{frame}

%%%%%%%%%%%%%%%%%%%%%%%%%%%%%%%%%%
 %%%%%%%%%%%%%%%%%%%%%%%%%%%%%%%%%%%%%%%%%

\frame{\frametitle{ DATA on  capital flows}
\begin{itemize}
\item Current account deficit of -4.4\% (Q3 2016) financed by foreigners acquiring South African assets
\end{itemize}
\begin{center}
\includegraphics[scale=0.8]{financial_accounts.png}
\end{center}
}


\frame{\frametitle{ DATA on  capital flows}


\section{   }
\begin{block}{ Association for Savings and Investments in South Africa ASISA}
ASISA is an industry body that publishes quarterly reports on the Collective Investment Schemes Industry in South Africa.
\begin{itemize}
\item CISs that are registered domestically
\item CISs that are marketed domestically (domiciled abroad)
\end{itemize}


\end{block}


\begin{block}{Morningstar}
Morningstar is an investment research and investment management firm headquartered in the United States.
\begin{itemize}
\item CISs that are domiciled and registered domestically
\item All other CISs globally
\end{itemize}

\end{block}
}

\frame{\frametitle{ DATA}
\section{ }
\includegraphics[scale=0.45]{growthCIS.png}
}

\frame{%\frametitle{Discussion and conclusion 2}
\begin{block}{}%[<+-| alert@+>]%<1-5>
	\centering
	{\LARGE THANK YOU!}
\end{block}
}

%%%%%%%%%%%%%%%%%%%%%%%%%%%%%%%%%%%%%%%%%%%%%%%%%%%%%%%%%%%%%%%%%%%%%%%%%%%%%%%%%%%%%%%%%%%%%%%%%%%%%%%%

%\bibliographystyle{Haba}
%\bibliography{Msc}
\end{document}
